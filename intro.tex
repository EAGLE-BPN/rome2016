\chapter{EAGLE: PAST, PRESENT AND FUTURE}

\section*{Past and Present}
Like any other European project, EAGLE had to be reviewed, at the end of each year, by the Project Officer of the European Commission and two external reviewers.\footnote{This texts reproduces without substantial changes the speech held on January 27th, at the opening of the EAGLE Final Conference.} Both in 2014 and in 2015, after 12 and 24 months respectively, our project was evaluated as ``excellent'', even if we were invited to consider some critical points. From the very formal point of view of the European Commission, this means essentially that a given budget was used and reported in the correct ways and time, that documents and items were delivered by the deadline, and that tasks and milestones declared in the Description of Work were achieved.
Actually we are approaching the final goal of the project, that is to make accessible to the public --- not only to scholars --- a huge amount of texts and images, with related metadata, pertaining to ancient Greek and Latin inscriptions. The technical aspects and the positive results are illustrated by Claudio Prandoni, EAGLE Technical Coordinator, in a dedicated presentation and panel (see the EAGLE panel p. \pageref{peaglepanel}, and the contributions by Mannocci p. \pageref{pCNR}, Prandoni p. \pageref{pXX}, Rocco p. \pageref{p28} and Vassallo-Damnjanovic p. \pageref{p44}). As Scientific Coordinator, I would like instead to make some more theoretical observations and share some thoughts on the significance of this ``excellent'' evaluation and the reasons that lay behind what we can call a successful case.

In fact, EAGLE is not only a project that is careful in spending money and sending deliverables and reports, ready to pass any review: EAGLE was born to be much more than this.


The possibility of free online access to all the Greek and Latin inscriptions of the ancient world has been a real need of the academic community for a long time. Our aim is to reach this goal not with the creation of a new, single database, but with the aggregation of the digital archives of different institutions around a common project and a common idea of what an inscription is, and how to read it; this idea was born after years of preliminary discussions and agreements that preceded and fostered the European project. Knowing clearly the need that we wanted to meet, and the way in which we wanted to do it, has surely helped us to focus our energies on the project’s main goals: the harmonization of very different materials and the creation of a new portal that could make them accessible through a search form specifically designed for inscriptions. We can say that both goals have been achieved, even if corrections and improvements are still possible and necessary. However, strange as this statement may seem, I don’t think that this is the real indicator of the success of the project. In my opinion, what actually shows that EAGLE is meeting a real need, is that our modus operandi is becoming an international standard: a larger and larger number of projects are using the controlled vocabularies that are one of the most interesting and immediately re-usable products of the harmonization job that EAGLE has done, are adopting an EAGLE compatible metadata model, and are sharing their content through the same system.


This scenario was already clear during the First EAGLE International Conference, which was held in Paris in autumn 2014, and it is now confirmed by the papers and posters selected to be presented in this volume, and by the huge networking activity of the project, which has widely enlarged the EAGLE consortium. Among the new partners --- too many to be listed here --- I would like to mention at least two examples that are --- for different reasons --- particularly important: the Inscriptions of Greek Cyrenaica (\url{https://igcyr.unibo.it/}) and the EPNet project (\url{www.roman-ep.net}), which is currently digitising the \emph{tituli picti} on the Roman amphoras from the Monte Testaccio in Rome. In the first case, one can clearly see the importance of a project dealing with inscriptions from a region corresponding to present-day Lybia: thanks to this digital archive, they will not only be better known, but also virtually preserved and protected against war damages and illegal commerce. In the second case, EPNet is an important step towards a digital archive for inscribed \emph{instrumentum}, a type of inscription that is fundamental to our knowledge of the ancient economy, but that is still lacking a policy of aggregation and harmonization of the existing digital resources.


This role as project of reference, of course, gives EAGLE a great responsibility, not only for the present, but also for the future, since now we must not only continue to meet the need for which the project was born, but also to show the direction in which we want to continue.

In this sense, and not by chance, I think, Euroepana has also shown an evolution in its policy: as has been recently stated by Joris Pekel during the final conference of the Athena Plus project (\url{http://www.athenaplus.eu/index.php?en/202/athenaplus-final-conference}), in the future Europeana will pay much more attention to the quality over the quantity of its digital items. This decision may have been made possible by the influence of projects like EAGLE, that are particularly careful about the curation of content and its usability in research activity.

The same observation can be made with respect to another aspect of the project. One of the main goals of the EAGLE project is to make inscriptions accessible not only to scholars, but also to a broader public, made up of students, teachers, tourists, and curious and interested people. To reach this audience it’s necessary to overcome the barriers represented by ancient languages and sometime ancient alphabets, epigraphic formulas and abbreviations, but also by the characteristics of traditional academic language and means of communication, using the huge potential of images, social media, and storytelling techniques. The results of this effort are particularly interesting: the realization of a mobile application using an image-based recognition system, the creation of the EAGLE MediaWiki platform to collect and organize thousands of translations in modern languages of epigraphic texts of varying complexity, a new storytelling application to illustrate the narrative content of many inscriptions, not to mention a virtual exhibition and a promotional video (see the EAGLE featured panel p. \pageref{peaglepanel}).

All this was included in the proposal submitted to the European Commission, but --- once again --- the fulfillment of promises is not the only indicator of the success of our project.  In this case too, I think that much more significant is what went beyond the promises: the unexpected, but no less interesting, developments. 

The possibility to have in a single online archive most of the existing translations of Greek and Latin inscriptions has raised a whole series of theoretical reflections about the often underestimated problems and difficulties that a translator has to face. On this subject, scientific contributions and practical solutions have been proposed (see, for example, the Guidelines for Translators in EAGLE MediaWiki: \url{http://www.eagle-network.eu/wiki/index.php/Guidelines_for_Translators}), but new questions still must be answered. And this shows that even a task born as a dissemination activity can successfully interact, if seriously undertaken, with research activity.

In the same way, the enormous work of enriching the content of the project with images has led to an effort to to clarify the different laws that in different countries of the European Union govern the use of photographs of cultural heritage for both educational and commercial purposes. In this framework, the EAGLE Consortium has shared the position of Europeana on the review of the EU copyright rules (\url{http://pro.europeana.eu/blogpost/a-first-glance-into-the-future-of-eu-copyright-reform}) and the re-use of digital images on the web, especially within projects related to the digitalization of cultural heritage. We also hope that in the future this assessment will be accepted by all the institutions that don’t yet recognize the civic value of scientific projects like ours.

In this field, too, the effort originally intended to enlarge the accessibility of the epigraphic material through the EAGLE portal is having --- and will have in the future --- interesting repercussions for scholars.

The importance of non textual elements for the correct and complete understanding of epigraphic messages has been recognized for some time. This implies the need not only to read, but also to look at the inscription. Not by chance, the theme chosen for the last International Conference of Greek and Latin Epigraphy, held in Berlin in 2012, was Öffentlichkeit --- Monument --- Text, and the same approach can be found in some of the papers presented in this volume, dealing with the relationship between form and content in epigraphic studies (see papers by Felle p. \pageref{p40}, Benefiel p. \pageref{p43}, Graham p. \pageref{p10}).


You can imagine how many possibilities in the field of palaeography and writing technique can be opened by the ability to search for ``similar images'' through the EAGLE portal. Once again, a project can be recognized as successful if it not only meets a present need, but also shows new directions for future research, that will benefit from a constant and closer relationship between texts and images.

Finally, let me say something that is not strictly epigraphic, but is no less important. Among all the things that I have learned during these three years of work as coordinator, there is the persuasion that all these results would have never been possible without the help of all the people who, in different ways and with different roles, are involved in the project. People first of all curious and keen to ask questions, ready to listen and observe, who don’t use problems as a pretext not to do things, but rather try to solve them. People, above all, able to connect ideas, places, projects, and other people in the awareness that every success is not a point of arrival, but part of a continuing journey.

We can learn something from this experience, not only for our present satisfaction but also as a suggestion for the future: even in the field of digital epigraphy we have to move, I think, towards a wider connection and interoperability of projects, allowing us not only to progressively fill the still existing gaps, but also to better use our human and financial resources.


In the XIXth century, the \emph{Corpus Inscriptionum Latinarum} would have never been completed without the huge net of collaborators, correspondants and scholars from the whole of Europe with whom Theodor Mommsen intensively exchanged letters and documents. In the same way, I think, the new frontier of epigraphy is the broadening of studies and research made more and more open, collaborative and constantly updated thanks to a clever use of technology and digital resources.

\sign{Silvia Orlandi}

\section*{Future}
There is no doubt that mine is the most difficult of the tasks assigned for this introduction. When we speak about past and present, we already know what has happened or what is happening. And this can also be a very pleasant task if, as in our case, there are many achievements and good results that can be praised.

On the contrary, it’s very different to speak about the future, that is something that has not happened yet, and that we don’t even know with certainty will ever happen. 

When I organized the XIth International Conference of Greek and Latin Epigraphy, which was held in Rome about 20 years ago, I asked Giancarlo Susini to open the conference, with the lecture ``Ten conferences plus one: the way of epigraphy''; on the opposite side, I asked Géza Alföldy to close the conference with a paper on ``The future of epigraphy'', and he, with the same embaressment that I have now, observed: ``As historians, we already have problems understanding the past. What could we say about the future?''

As a matter of fact, the future doesn’t lean on solid, measurable facts: it’s the reign of the unknown, in which one projects rational and irrational desires, mixed with fears or even anguish. But it’s also a temporal space in which many of our previsions are going to be swept away by facts that nobody had foreseen or even imagined.

It would have seemed more logical for the future of the project to be presented by a young person, who will have the time to experience it, and not by an old man, who will never see it. By the way, I’m rather upset when I see the media invaded every day by some old men --- I hope not to be one of them --- who, after having had a long time to express their opinion, and having failed, think that they still have the right to give rules for the future, while --- maybe - it’s the moment to open the floor to others.

But let’s speak of lighter things!

I don’t think that you want me to become a ``futurologist'' and to make predictions of what is going to happen in the world, or, more modestly, in EAGLE, in a more or less distant future. But we can’t do without the future.

Every action presupposes a future, and this feeling leads our steps and gives them a meaning. But we can also say that the future is nothing but the present of yesterday. To speak about the future of EAGLE, therefore, there is no need to imagine complex scenarios for the coming years. We can just ask ourselves: What am I going to do tomorrow? In English, there is a convenient distinction between the simple future, that is used for actions that happen spontaneously, and the intentional future, that is used for actions that are consequences of a plan, an intention. That’s what we should talk about now, but we can’t do that without considering:
\begin{description}
\item[a] What was our original goal and how much of it has been achieved
\item[b] What has worked well and is to be kept, and what, according to this experience, should be improved
\item[c] What has been set aside or not originally included, but should be planned for the future
\end{description}

Since the beginning, I’ve thought that my role in the project was not to define every detail, but to establish some key points, and check that they would not be forgotten or changed. That’s why today I would like not to illustrate every aspect of the project, but just to say a couple of words about the points already mentioned.

``Old fashioned'' EAGLE was born as a federation of databases with the goal of ``recording all Greek and Latin inscriptions older than the VIIth century AD, according to the best existing edition, possibly checked and improved, along with some fundamental data and images'' (see the documents collected in \url{http://www.eagle-eagle.it/Italiano/documenti_it.htm}). We can say that this goal has been adopted by the ``new version'' EAGLE, too. But, due to a difficult coordination between Greek and Latin epigraphers (an old problem…) most attention has actually been paid to Latin inscriptions. We have known since the beginning that the complete recording of all known inscriptions was practically impossible. Nevertheless, we can probably state that in a short time the number of inscriptions searchable through the EAGLE portal will reach about 550.000, thanks to the enlargement of the consortium and to the enormous work of harmonization and disambiguation of content, and above all thanks to the inclusion of the epigraphic texts put at our disposal by Manfred Clauss and his database (I would like to thank Pietro Maria Liuzzo for this and other information).

It’s not all that we need (I’ll come back on this point later) and, above all, the metadata set of the texts and their degree of elaboration is not homogeneous in all the records. But nobody can deny that the original plan has been mostly fulfilled, and looking at what has been done we can be rather confident in the work that awaits us in the future.

Our tasks for tomorrow are the subject of my second point, concerning what to keep, and what to change according to our past experience. In my opinion, and --- as it seems --- according to the European Commission too, the general structure of the project, what we can call its philosophy, has been successfully tested --- as the facts testify --- and should, therefore, be maintained. 

The philosophy is based on two fundamental principles:
\begin{enumerate}
\item The first is common to every project that aims to be a scientific research project, and it’s the awareness that we do not have definitive solutions, but just hypotheses, which must always be checked because every attempt involves the possibility of errors that can and must be corrected. The large number of changes and improvements made during the project are not a proof of weakness, but a sign of its strength.
\item The second point is the clear need for as wide a collaboration as possible, to ensure not only a large quantity of data, but above all a high quality of content, checked by experts in different geographic regions and thematic fields. This aspect has been particularly curated in these last years, so that the number of institutions and single content providers has been greatly increased. In this way, and thanks to great technical work including the fundamental creation of controlled vocabularies, the EAGLE portal now gives access to many different databases, originally independent and with different characteristics and purposes. In my opinion, this is the path to follow in the future as well, in order to face and solve bigger and bigger problems.
\end{enumerate}

EAGLE looks different from other similar projects because, since the beginning, it has paid much attention not only to the quality of information, but also to the combination of textual and non-textual data, according to the current definition of an inscription as an inscribed monument. Moreover, in recent years, we have seen a huge increase in the visual documentation available online, and now EAGLE includes about 250.000 digital photos of inscriptions. This is another aspect that should be maintained, and --- possibly - even strengthened. In fact, thanks to projects like ours, the problems related to the legal treatment of images of cultural heritage seem to be at the moment under discussion, both nationally and internationally (see paper by Modolo p. \pageref{p38}). Silvia Orlandi is also right to underline how important the inclusion of images of inscriptions and the technical possibility to search them will be for paleographic studies. I have recently discussed with Silvia Evangelisti how to improve the ``\emph{scriptura}'' field of EDR with more detailed information about writing techniques, materials and tools. But the analysis of the graphic forms of inscriptions has not yet been adequately confronted, as it still lacks the contribution of professional paleographers. Maybe this conference will give us the chance to begin this kind of discussion.

Speaking about old goals to be taken up again, or new goals to be achieved, I have to come back to the problem of Greek inscriptions and the inscribed \emph{instrumentum}. They are both essential components of our research field, with a number of independent digital repositories, but since the beginning we have had problems with their inclusion in the project. At the moment, EAGLE includes more than 7000 Greek inscriptions, while a specific commission of the International Association of Greek and Latin Epigraphy is currently working on the digital \emph{instrumentum}. Now it seems that, after a long period of inactivity, we are having in recent years an awakening on both sides, and particularly on the side of Greek epigraphy, thanks above all, to the \emph{Inscriptiones Graecae}. Now it’s time to aggregate this fundamental material too, with the help of new technologies. I would just suggest that we first reach a preliminary agreement among the participants using past experience rather than starting from scratch both in terms of base requirements, and of open and flexible structure.

All this is about the future: not a vague and undefined future, but a very positive future, modeled according to our past and present plans, constantly checked and renewed. But at least part of the future doesn’t depend on us.

For example: Will there still be somebody who will trust and fund projects like ours?

And will there still be, in Italy and elsewhere, enough scholars, old and young, adequately educated and motivated, who will take care of it?

The tendencies that we can see in Italian and European policy in the field of culture and the university, or at least some of them, could lead us to pessimism, but I don’t think that we should give up. First of all because there are also signs of hope. The European Union, for example, with other important national institutions whose moral and financial support should never be forgotten --- has trusted and supported the proposal that we submitted three years ago, and, during annual reviews, has appreciated the way in which the project has been carried out under the guidance of Silvia Orlandi and her many collaborators. Why should we exclude the possibility that something similar will happen again in the coming years? I continue to believe that EAGLE is not an ordinary project and that its cultural importance, both for the scientific community and for civic life, will be adequately recognized.

Moreover, my confidence is increased by the enthusiasm with which so many young people have taken part in the project, giving and receiving so much, not only on the professional and cultural side, but also in terms of education and ethics, learning to work not only for themselves, but also for others.

We live in a time of very quick changes --- and this is even more true for a university, where the population of students is almost completely renewed every five years. Therefore, to foresee the worst is not necessarily more realistic than having some hopes.

Anyway, I think that not only a scholar, but any person who believes in something, should not give up only because of a mere calculation of probability: luckily we are not working only for the market, so that we can leave to others the task of dealing with risks and probabilities.


\sign{Silvio Panciera}