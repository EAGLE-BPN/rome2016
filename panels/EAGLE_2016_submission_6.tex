% !TEX encoding = UTF-8 Unicode
% !TEX program = pdflatex
% !TEX spellcheck = en_US


% In order to correctly compile this document,
% execute the following commands:
% 1. pdflatex
% 2. pdflatex
% 3. pdflatex



\documentclass[amsthm,ebook]{saparticle}

% IF YOU USE PDFLATEX
\usepackage[utf8x]{inputenc}
% if you write in english and in greek
\usepackage{ucs}
\usepackage[greek,english,francais]{babel}
\languageattribute{greek}{polutoniko}

% IF YOU USE XELATEX
%\usepackage{polyglossia}
% if you write in italian
%\setmainlanguage{italian}
% If you want put some ancient greek:
%\setotherlanguage[variant=polytonic]{greek}
%\newfontfamily{\greekfont}[Ligatures=TeX]{Palatino Linotype}

% dummy text (remove in a normal thesis)
% remove if not necessary
\usepackage{siunitx}
%Natbib for bibliography management
\usepackage[authoryear]{natbib}
% custom commands
\newcommand{\bs}{\textbackslash}
\usepackage{xtab}
%%%%%%%%
%TITLE:%
%%%%%%%%

\title{Assessing the Role of Digital Libraries of Squeezes in Epigraphic Studies. Digitization, visualization, and metadata}

\author[Flo]{Eleni Bozia\corref{first}}

\address[Flo]{University of Florida}
\cortext[first]{Corresponding author. Email: bozia@ufl.edu}

%\usepackage{longtable}
\begin{document}

\maketitle

\keywords{Squeezes, Online Libraries, 3D Visualizations, Contextuality}

\section{Description}
This panel discussed issues and research questions regarding the use and efficiency of digital libraries of squeezes of inscriptions—structure of online libraries, nature of the metadata, 3D digitization, and visualization methods. The panel that consisted of epigraphists, digital epigraphists, and computer engineers engaged in a dialogue, addressing the above issues from different perspectives.

Digital libraries of squeezes have become an integral part in epigraphic studies and research. Whether it is the issue of collecting data for more efficient use, increasing their accessibility, or simply digitally preserving them, digital libraries are a new scholarly medium. Such projects confront the challenge of having to determine the types of data that should be included—traditional information, digital metadata, formats and tools for articulating all the specificities of the metadata—as well as decide whether the type of the project should predetermine the ontology of the data.

The aforementioned primary issue begs also the question of the need for interoperability of e-libraries in an attempt to combine, contrast, and comparatively appraise and study the artifacts themselves and their metadata. Thus far several projects, including but not limited to the EAGLE consortium, the Center for Epigraphical and Palaeographical Studies, the Aleshire Collection at the University of California, Berkeley, and the US Epigraphy Project among others feature digital libraries of squeezes; however, how should one proceed about perusing them? Issues that need to be addressed include: whether their use would be facilitated by means of homogeneity of the libraries, or whether the variegated nature of the material could potentially increase the number of users and the amount of information that is available.

However, one should consider the limitations of e-libraries that focus solely on squeezes. When discussing the nature of the digital libraries’ content and the implication that the metadata may have in the interpretation of a squeeze, we need to consider its contextuality—whether the absence of the inscription bearer from a squeeze’s record may prove reductive with regards to the holistic approach to the text. However, notwithstanding that high-resolution digitization of squeezes opens the possibilities for enhanced study on a level that until now could not be achieved, techniques that focus solely on the digitization of the squeezes usually fail to deal with the inscription bearer. Furthermore, there are other types of data—geo-spatial, prosopographic, linguistic—that would not only enhance our understanding of any one particular squeeze, but also contribute to our overall apprehension of classical, archeological, and epigraphic studies. This panel discussed how the digital records of squeezes could be augmented to encapsulate their contextual information. 

Questions that were addressed:

\begin{enumerate}
\item How to preserve the traditional nature of the data—specificity, terminology—while retaining the possibility of keyword search?

\item Have we started thinking/searching/researching differently. Is it the database that determines the type of research that one conducts?

\item Do we create authenticity via our digitization selections? In turn does this mean that we validate?

\item How can computer scientists and humanists communicate to find a common point of reference between creating efficient algorithms and databases while retaining the nature of epigraphic and archaeological studies?

\item TEI has created a common form of expression and advanced the homogeneity of data. Could we proceed with a similar trajectory for other forms of data? 

\item Would such a scenario also resolve the issue of contextuality. One database need not be an artifact’s concordance, but interlinked databases could preserve their individuality while advancing search possibilities?

\item What are other aspects of the artifact (inscription bearer etc.) that could enhance the record of the squeeze?
 
\item What kind of information would produce a more holistic record for the squeezes.
\end{enumerate}


	Eleni Bozia gave a brief introduction to the history of online databases of squeezes. She presented the type of data and metadata that they favor and proceeded to furnish the problems that surface when research groups do not opt for interlinked databases or structure the metadata according to the tools available, rather than according to the research needs and the nature of the database. Prof. Bozia also argued in favor of flexibility of databases and projects with the intent to consider the needs of each database rather than favor an existing ontological structure that restricts the content, and contextuality of the database, thus limiting its effectiveness.
Adeline Levivier presented the collaboration between Université Lyon 2, UMR HiSoMA \& Ecole française d’Athènes and the Digital Epigraphy and Archaeology Project at the University of Florida. The project invoves the digitization of approximately 9000 squeezes in 3D as well as the creation of an online library with metadata of the collection. The squeezes date from Archaic Greek period to the Roman Imperial era and originate from Delos, Thasos, Delphi, and Asia Minor. The group pursues dematerialization for long-term preservation, remote consultation of data, standardization for interoperability with other resources related to cultural heritage and digital epigraphy, and creation of a new resource for research: a big data corpus to  be queried by advanced systems. Issues of selection were brought forward—how do research groups make choices— as well as the usability of 3D models of squeezes that give the option of advanced visual manipulations and automatic measurements. The major matter in question that the group intends to address in this project is the need to create a database that fits the researcher’s needs instead of merely reusing existing expertise and ontology. 

	Manuel Sánchez and Jose Pablo Su\'arez-Rivero advanced the discussion by presenting options for the enhancement of the digital libraries of squeezes. They presented their project titled Epigraphia 3D in which they perform 3D digitization of the entire artifact/inscription bearer. This admittedly provides a more holistic record of the squeeze and a context for the inscription itself. Profs. Sánchez and Su\'arez-Rivero presented their methodology that employs photogrammetry, their results, and their online database with the artifacts. An issue that was brought forward is how one can combine within the same project a detailed 3D digitization of the letters of the inscription/squeeze as well as a large-scale artifact/inscription bearer. The advantages, however, of a more complete record of the artifact and the squeeze became apparent. 

Finally, Angelos Barmpoutis concluded the panel with a discussion on the Digital Epigraphy Project and its metadata perspective, arguing in favor of a collected record as well as the flexibility to incorporate fields and metadata particularities as needed in its case. He discussed the need to treat both the physical squeeze as well as the digital artifact as objects that require metadata. He presented the sandbox upon which the Digital Epigraphy project functions with regards to the addition of metadata information. Prof. Barmpoutis also discussed the significance of considering every user’s unique perspective should one wishes to increase the usability and importance of their research database. 

In conclusion, the panel discussed theoretical issues that derive from the exigency to communicate and consider the advancements of technology alongside the needs of traditional research and presented current projects, databases, and actual case scenaria that deal with such issues. The intent of this multifarious cohort of scholars was to emphasize the importance of transdisciplinary collaboration and the pivotal role of the individual researcher/user. The panel considered ways to utilize traditional knowledge with the advantageous flexibility of digital tools that will ultimately not only facilitate, but also ultimately enhance epigraphic studies.

\section{Panelists}
\begin{description}
\item[Michèle Brunet]  \emph{Université Lyon 2, UMR HiSoMA \& Ecole française d’Athènes  Email: Michele.Brunet@univ-lyon2.fr} Prof. Brunet has been the Chair of Greek and Latin Epigraphy at the Université Louis Lumière Lyon 2 since 2006. She has been a member of l'École Normale Supérieure de Paris (1979-1984) and l'École française d'Athènes (1984-1988).

\item[Adeline Levivier] \emph{Université Lyon 2, UMR HiSoMA \& Ecole française d’Athènes Email: adeline.levivier@gmail.com} Adeline is a doctoral candidate in l’École française d'Athènes - Université Lyon 2. She works on the digitization of squeezes. She collaborated on the project ANR E-pigramme (Épigraphie et Muséographie - Édition numérique et valorisation de la Collection des inscriptions grecques du Musée du Louvre). She is currently the program manager of the project E-stampages (Numérisation et diffusion web en 3D des collections de l'UMR 5189 HiSoMA et de l'Ecole française d'Athènes).


\item[Manuel Ramírez Sánchez] \emph{Universidad de Las Palmas de Gran Canaria Email: manuel.ramirez@ulpgc.es} Prof. Sánchez is a Professor of Historiographic Sciences and Techniques at the Department of Historical Sciences. He works on the 3D digitization and advanced visualizations of inscriptions and has published extensively in the area of inscriptions from Ancient Hispania.

\item[Jose-Pablo Suárez-Rivero] \emph{Universidad de Las Palmas de Gran Canaria Email: Josepablo.suarez@ulpgc.es} Prof. Suárez-Rivero is the Director of Política Informática at the Cartography and Graphic Engineering Department. He has a Ph.D. in Applied Mathematics and works on mesh generation, algorithms and data structures, and computational geometry. 

\item[Angelos Barmpoutis] \emph{University of Florida Email: angelos@digitalworlds.ufl.edu} Prof. Barmpoutis is an Associate Professor in the On-line Institute and the Digital Worlds Institute at the University if Florida. He is also the coordinator of research and technology at the Institute and affilitate faculty at the Computer Science and Engineering Department. His research focuses on interdisciplinary applications of computer science to the service of broad areas of learning and training.  
\end{description}



\end{document}
