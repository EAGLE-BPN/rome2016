% !TEX encoding = UTF-8 Unicode
% !TEX program = pdflatex
% !TEX spellcheck = en_US


% In order to correctly compile this document,
% execute the following commands:
% 1. pdflatex
% 2. pdflatex
% 3. pdflatex



\documentclass[amsthm,ebook]{saparticle}

% IF YOU USE PDFLATEX
\usepackage[utf8x]{inputenc}
% if you write in english and in greek
\usepackage{ucs}
\usepackage[greek,english,francais]{babel}
\languageattribute{greek}{polutoniko}

% IF YOU USE XELATEX
%\usepackage{polyglossia}
% if you write in italian
%\setmainlanguage{italian}
% If you want put some ancient greek:
%\setotherlanguage[variant=polytonic]{greek}
%\newfontfamily{\greekfont}[Ligatures=TeX]{Palatino Linotype}

% dummy text (remove in a normal thesis)
% remove if not necessary
\usepackage{siunitx}
%Natbib for bibliography management
\usepackage[authoryear]{natbib}
% custom commands
\newcommand{\bs}{\textbackslash}
\usepackage{xtab}
%%%%%%%%
%TITLE:%
%%%%%%%%

\title{Assessing the Role of Digital Libraries of Squeezes in Epigraphic Studies. Digitization, visualization, and metadata}

\author[Flo]{Eleni Bozia\corref{first}}

\address[Flo]{University of Florida}
\cortext[first]{Corresponding author. Email: bozia@ufl.edu}

%\usepackage{longtable}
\begin{document}

\maketitle

\keywords{Squeezes, Online Libraries, 3D Visualizations, Contextuality}

\section{Description}
This panel will discuss issues and research questions regarding the use and efficiency of digital libraries of squeezes of inscriptions—structure of online libraries, nature of the metadata, 3D digitization, and visualization methods. The panel that consists of epigraphists, digital epigraphists, and computer engineers will engage in a dialogue, addressing the above issues from different perspectives.

Digital libraries of squeezes have become an integral part in epigraphic studies and research. Whether it is the issue of collecting data for more efficient use, increasing their accessibility, or simply digitally preserving them, digital libraries are a new scholarly medium. Such projects con- front the challenge of having to determine the types of data that should be included—traditional information, digital metadata, formats and tools for articulating all the specificities of the metadata—as well as decide whether
 the type of the project should predetermine the ontology of the data.

The aforementioned primary issue begs also the question of the need for interoperability of e-libraries in an attempt to combine, contrast, and comparatively appraise and study the artifacts themselves and their metadata. Thus far several projects feature digital libraries of squeezes; howe- ver, how should one proceed about perusing them? Whether this would be facilitated by means of homogeneity of the libraries, or whether the variegated nature of the material could potentially increase the number of users and the amount of information that is available.

However, one should consider the limitations of e-libraries that focus solely on squeezes. When discussing the nature of the digital libraries’ content and the implication that the metadata may have in the interpretation of a squeeze, we need to consider its contextuality—whether the absence of the inscription bearer from a squeeze’s record may prove reductive with regards to the holistic approach to the text. However, notwithstanding that high-resolution digitization of squeezes opens the possibilities for enhanced study on a level that until now could not be achieved, techniques that focus solely on the digitization of the squeezes usually fail to deal with the inscription bearer. This panel will discuss how the digital records of squeezes could be augmented to encapsulate their contextual information.




\section{Panelists}
\begin{description}
\item[Michèle Brunet]  \emph{Université Lyon 2, UMR HiSoMA \& Ecole française 
d'Athènes  Email: Michele.Brunet@univ-lyon2.fr} Prof. Brunet has been the Chair of Greek and Latin Epigraphy at the
Université Louis Lumière Lyon 2 since 2006. She has been a
l'École Normale Supérieure de Paris (1979-1984) and l'École française d'Athènes (1984-1988), She received her HDR on the Study of Antiquity at the Sorbonne. She taught Archaeology and Greek Art a Bordeaux III and Paris I Panthéon-Sorbonne. She also served as the Director of Ancient Studies at l’École française d'Athènes of member.
\item[Adeline Levivier] \emph{Université Lyon 2, UMR HiSoMA & Ecole française d’Athènes Email: adeline.levivier@gmail.com} Adeline is a doctoral candidate in l’École française d'Athènes - Université Lyon 2. She works on the digitization of squeezes, and her thesis is entitled ‘Recherches sur l'écriture grecque à partir des collections d'estampages d'inscriptions.’ She collaborated on the project ANR E-pigramme (Épigraphie et Muséographie - Édition numérique et valorisation de la Collection des inscriptions grecques du Musée du Louvre). She is currently the program manager of the project E-  stampages (Numérisation et diffusion web en 3D des collections de  l'UMR 5189 HiSoMA et de l'Ecole française d'Athènes).


\item[Manuel Ramírez Sánchez] \emph{Universidad de Las Palmas de Gran Canaria Email: manuel.ramirez@ulpgc.es} Prof. Sánchez is a Professor of Historiographic Sciences and Techniques at the Department of Historical Sciences. He works on the 3D digitization and advanced visualizations of inscriptions and has published extensively in the area of inscriptions from Ancient Hispania.

\item[Jose-Pablo Suárez-Rivero] \emph{Universidad de Las Palmas de Gran Canaria Email: Josepablo.suarez@ulpgc.es} Prof. Suárez-Rivero is the Director of Política Informática at the Cartography and Graphic Engineering Department. He has a Ph.D. in Applied Mathematics and works on mesh generation, algorithms and data structures, and computational geometry.

\item[Angelos Barmpoutis] \emph{University of Florida Email: angelos@digitalworlds.ufl.edu} Prof. Barmpoutis is an Associate Professor in the On-line Institute and the Digital Worlds Institute at the University if Florida. He is also the coordinator of research and technology at the Institute and affiliate faculty at the Computer Science and Engineering Department. His research focuses on interdisciplinary applications of computer science to the service of broad areas of learning and training.
\end{description}


\section{Notes}
The intent of this panel of a multifarious cohort of scholars is to pose questions that have derived from the advancements of technology and digital epigraphy research. The panel will consider possible answers and ways to utilize traditional knowledge with the advantageous flexibility of digital tools that will ultimately not only facilitate, but also enhance epigraphic studies.
\end{document}
