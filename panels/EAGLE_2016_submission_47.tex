% !TEX encoding = UTF-8 Unicode
% !TEX program = pdflatex
% !TEX spellcheck = en_US


% In order to correctly compile this document,
% execute the following commands:
% 1. pdflatex
% 2. pdflatex
% 3. pdflatex



\documentclass[amsthm,ebook]{saparticle}

% IF YOU USE PDFLATEX
\usepackage[utf8x]{inputenc}
% if you write in english and in greek
\usepackage{ucs}
\usepackage[greek,english]{babel}
\languageattribute{greek}{polutoniko}

% IF YOU USE XELATEX
%\usepackage{polyglossia}
% if you write in italian
%\setmainlanguage{italian}
% If you want put some ancient greek:
%\setotherlanguage[variant=polytonic]{greek}
%\newfontfamily{\greekfont}[Ligatures=TeX]{Palatino Linotype}

% dummy text (remove in a normal thesis)
% remove if not necessary
\usepackage{siunitx}
%Natbib for bibliography management
\usepackage[authoryear]{natbib}
% custom commands
\newcommand{\bs}{\textbackslash}
\usepackage{xtab}
%%%%%%%%
%TITLE:%
%%%%%%%%

\title{Epigraphic Echoes in Epigrams}

\author[CLUJ]{Marion Lam\'e\corref{first}}

\address[CLUJ]{Centre Camille Jullian, MMSH, CNRS, France}

\cortext[first]{Corresponding author. Email: mlame@mmsh.univ-aix.fr}

%\usepackage{longtable}
\begin{document}

\maketitle

\keywords{interoperability between editions, epigram, traditional and digital epigraphy, cross-disciplinary studies of inscriptions}


\section{Description}

For several years, the Memorata Poetis project (PRIN 2010/11) aims at studying the intertextual relationships between literary and epigraphic epigrams in several languages (Ancient Greek, Latin, Italian, Arabic and English). In order to identify some patterns that echo across space, time, culture and languages, one of the main activity consists in manually identifying Themes \& Motifs within the corpus of poems. Such Themes \& Motifs are organized, for now, as a hierarchical index of metadata that are associated with the text of the poem only. However, part of this corpus is composed of epigraphic primary sources. This means that such epigrams were set up in a deliberate context of communication and that part of their message is based on linguistic and non linguistic elements of their contextual, textual and writing systems. Some examples are iconographic programs and relationships with other poems, type of verses, inscriptions, themes and motifs (ex. AE, 1967, 85 ; De Hoz, 2014, No 355, Banti, No 5 and 51). In this panel, participants would like to discuss convergences and divergences between textual editing that tends to follow philological methods, and document editing, that focuses on the unicity of the materiality of the carrier (Pierazzo, 2015) and that appears more adapted to epigraphic edition. Both types of edition would facilitate cross-disciplinary studies of epigraphic epigrams. Panelists will focus on how connecting both types of edition thanks to digital technologies, i.e. integrating digital edition of epigraphic epigrams in the wider context of digital philology on one hand (e.g.: giving access to high quality images thanks to computer graphics tools) and in the one of public history on the other hand (e.g.: crowdsourcing with historical method).

DE HOZ M. 2014, Inscripciones griegas de España y Portugal, Madrid. 
PIERAZZO E. 2015, Digital Scholarly Editing: Theories, Models and Methods, Farnham: Ashgate.
BANTI, O. 2000, Monumenta epigraphica pisana saeculi XV antiquiora, Pisa.

\section{Panelists}
\begin{description}
\item[Dr. Marion Lam\'e]\emph{Centre Camille Jullian, MMSH, CNRS, France Email: mlame@mmsh.univ-aix.fr} Collaborates with the Laboratorio di Cultura Digitale of the University of Pisa. She has worked and works on several digital epigraphic projects (IGLouvre, TSS, Memorata Poetis...) and collaborates to collective dissemination of digital practices (EpiDoc). Her researches are dedicated to Digital Epigraphy applied especially to complex epigraphical situation and device (multilingualism, multialphabetism...).

\item[Pr. Paolo Mastrandrea] \emph{University of Venice Email: mast@unive.it}
P. Mastandrea is specialized in Latin Philology. He is the Project Coordinator of important national projects in digital humanities: Memorata Poetis and Musisque Deoque.


\item[Pr. Flavia De Rubeis] \emph{University of Venice.
Email: flavia.derubeis@unive.it} Specialized in Latin Paleography and is in charge of the epigraphic epigrams of the Middle Ages in Memorata Poetis Project.

\item[Dr. Mia Trentin] \emph{University of Venice Email: trentin.mia@unive.it} Mia Trentin is specialized in Medieval Epigraphy and Latin Palaeography, focusing on medieval ways of communication and expression through the analysis of historical graffiti writing. She collaborates with prof. Flavia De Rubeis for the project Memorata Poetis.

\item[Dr. Massimo Manca] \emph{University of Torino
Email: massimo.manca@unito.it} Researcher specialized in Latin Language and Literature, his interests are Latin literature in Late antiquity, didactics of Ancient Greek and Latin, training of teachers, digital tools 
for classics.

\item[Pr. Alfredo Morelli] \emph{Università degli Studi di Cassino Email: alfmorel@unicas.it} Professor of Latin language and literature at the University of Cassino (Italy), he works on the history of Greco-Roman literary and epigraphic epigram of Hellenistic period until the age of Martial, the Roman elegy at the time of Augustus and the tragedies of Seneca. He has been active for years on the Carmina Latina Epigraphica in cooperation with Epigraphic Database Roma.

\item[Pr. Enrica Salvatori] \emph{University of Pisa
Email: enrica.salvatori@unipi.it} Specialized in Medieval history and Director of the Laboratorio di Cultura Digitale of the University of Pisa. She works in the digitization process of inscriptions of Pisa, included epigrams, and in projects of public history concering the valorization of epigraphs.
[\url{http://pisaeislam.humnet.unipi.it/} - \url{http://epigrapisa.humnet.unipi.it/}]





\end{description}


\section*{Acknowldgements}
Eleorona Santin for her suggestions while preparing this panel.

\end{document}

