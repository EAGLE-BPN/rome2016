% !TEX encoding = UTF-8 Unicode
% !TEX program = pdflatex
% !TEX spellcheck = en_US


% In order to correctly compile this document,
% execute the following commands:
% 1. pdflatex
% 2. pdflatex
% 3. pdflatex



\documentclass[amsthm,ebook]{saparticle}

% IF YOU USE PDFLATEX
\usepackage[utf8x]{inputenc}
% if you write in english and in greek
\usepackage{ucs}
\usepackage[greek,english,francais]{babel}
\languageattribute{greek}{polutoniko}

% IF YOU USE XELATEX
%\usepackage{polyglossia}
% if you write in italian
%\setmainlanguage{italian}
% If you want put some ancient greek:
%\setotherlanguage[variant=polytonic]{greek}
%\newfontfamily{\greekfont}[Ligatures=TeX]{Palatino Linotype}

% dummy text (remove in a normal thesis)
% remove if not necessary
\usepackage{siunitx}
%Natbib for bibliography management
\usepackage[authoryear]{natbib}
% custom commands
\newcommand{\bs}{\textbackslash}
\usepackage{xtab}
%%%%%%%%
%TITLE:%
%%%%%%%%

\title{EAGLE Featured Panel}

\author[ISTI]{Vittore Casarosa\corref{first}}

\address[ISTI]{Istituto di Scienza e Tecnologie
dell’Informazione ``A. Faedo''}
\cortext[first]{Corresponding author. Email: casarosa@isti.cnr.it}

%\usepackage{longtable}
\begin{document}

\maketitle


\section{Description}
The EAGLE project is really a multidisciplinary project,
which can be looked at from three different perspectives,
each one with many different facets.
\begin{itemize}
\item The epigraphy dimension, (inscriptions, transcriptions,
critical editions, translations, historical context,
geographical context, materials)
\item  The technology dimension (data model, data aggregation,
search engine, 3D representations, disambiguation,
image recognition)
\item  The application dimension (mobile application,
storytelling, virtual reality, epigraphy in schools)
\end{itemize}


In this panel, we will touch on some of those facets,
focussing more on the ones that are not going to be
presented during the conference. Our aim is to stimulate
some discussion and exchange of ideas, to try and
understand what we have missed, what could have been
done better, and what you (the audience) see as the
strong points of EAGLE.

\section{Panelists}
\begin{description}
\item[Anita Rocco] EDB 2.0
\item[Valentina Vassallo] 3D epigraphy
\item[Andrea Mannocci] data quality
\item[Francesco Mambrini] storytelling
\item[Giuseppe Amato] mobile application
\item[Sorin Hermon] virtual exhibition
\item[Luca Giberti] promotional video
\end{description}



\end{document}
