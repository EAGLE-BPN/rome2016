% !TEX encoding = UTF-8 Unicode
% !TEX program = pdflatex
% !TEX spellcheck = en_US


% In order to correctly compile this document,
% execute the following commands:
% 1. pdflatex
% 2. pdflatex
% 3. pdflatex



\documentclass[amsthm,ebook]{saparticle}

% IF YOU USE PDFLATEX
\usepackage[utf8x]{inputenc}
% if you write in english and in greek
\usepackage{ucs}
\usepackage[greek,english]{babel}
\languageattribute{greek}{polutoniko}

% IF YOU USE XELATEX
%\usepackage{polyglossia}
% if you write in italian
%\setmainlanguage{italian}
% If you want put some ancient greek:
%\setotherlanguage[variant=polytonic]{greek}
%\newfontfamily{\greekfont}[Ligatures=TeX]{Palatino Linotype}

% dummy text (remove in a normal thesis)
% remove if not necessary
\usepackage{siunitx}
%Natbib for bibliography management
\usepackage[authoryear]{natbib}
% custom commands
\newcommand{\bs}{\textbackslash}

\usepackage{url}
\def\UrlBreaks{\do\/\do-}
%%%%%%%%
%TITLE:%
%%%%%%%%
\title{Free Reproduction of Cultural Heritage for Research Purposes:
Regulatory Aspects and New Prospects for Project EAGLE}
\author[free]{Mirco Modolo\corref{first}}
\address[free]{Free lance archaeologist and archivist}
\cortext[first]{Corresponding author. Email: mircomodolo@gmail.com}
\date{2015-11-20}
\begin{document}

\maketitle
\begin{abstract}
This paper traces the genesis and activity of the ``Fotografie libere per i Beni Culturali'' movement promoting free
remote reproduction, and reuse for scientific purposes, of manuscripts and antique volumes conserved in Italian
archives and libraries. Simultaneously, evidence highlights the advantages of deregulating cultural heritage images for
historical research, in particular for the study of antique epigraphy developed by the EAGLE project.
\end{abstract}
\keywords{Archives, Libraries, Manuscripts, Reproductions, Creative Commons}

\section{Free reproduction of cultural heritage outlined in the ``Art Bonus'' decree}


On the first of June 2014, the `Art Bonus' decree came into effect, sanctioning the free reproduction of all cultural
heritage for scientific purposes (even reproduction carried out at a distance). It represented a new first for
historical research:\footnote{Decreto-legge 31 maggio 2014 n. 83, art. 12.} numerous archives and libraries opened
and made free of charge the reproduction of manuscripts and historical volumes with a user’s camera. This provided a
clear advantage for historical research, particularly epigraphy, which considers antique manuscripts from the 15th and
19th centuries an important source for the study of Latin and Greek inscriptions. 

As a direct consequence of the Art Bonus decree, some institutions, including the State Archives of Florence (Archivio
di Stato di Firenze), allowed users to make digital copies of records in consultation. Unfortunately, just one month
later, a restrictive amendment modified the original text of the law, explicitly excluding printed books, manuscripts
and archival documents from the liberalisation.\footnote{Legge 29 luglio 2014, n. 106} Many scholars’ surprise was
overtaken by irritation, driven by an apparent paradox: museum photography remained free for the tourist, yet the
scholar engaged in research activity in archives and libraries would not reap the benefits of
liberalisation.\footnote{\citet{modolo_il_2014}; \citet{brugnoli_ancora_2014}. }




\section{Current state of the problem}

Excluding bibliographic and archival records from liberalisation has effectively restored the previous regime, still in
force today: taxing images taken with a personal digital device (in the institutions still allowing this freedom), or
an outright ban on use of a personal device – requiring the paid use of a photographic service. Rates for using a
personal device are the most varied (on average 3 Euro per unit), while some archives require a rental fee for making
reproductions on a personal device in a specialised room. Thus an extreme variety of tariffs emerges, commoditising
research to the detriment of those who contribute to enhancing documentary heritage through their own study.
Furthermore this is in direct contrast with the Italian Constitution which, according to articles 33 and 9, requires
the Republic not only to guarantee free research, but also to actively promote it. An effective ``tax'' on reproduction
with a personal device becomes a research obstacle; it is absurd to expect a fee for a service provided autonomously by
the researcher himself. 

Regarding `reuse' of the image, publication for scientific purposes remains subject to a precise and formal request to
have `publication authorised' granted on stamped paper, hugely time consuming for administrator as well as applicant.
The payment of fees for use (`canoni di utilizzo' or royalties) is excluded instead for the publication of images of
cultural heritage (such as archival documents, ancient printed books or ancient inscriptions) in scholarly books or
papers with a cover price of less than 77 Euro and up to 2000 printed copies (as per circular 21/2005 of the `Ministero
dei Beni e delle Attività Culturali e del Turismo').




\section{Traditional objections to liberalisation of reproductions}



The principal objections opposing the liberalisation of reproductions of bibliographic and archival records, and which
must be addressed here, regard both `conservationist' as well as derivational `improper use' risks for the images.

The protection of archival material is one of the principal arguments against free reproduction. Indeed archivists at
the State Archives of Florence (Archivio di Stato di Firenze) directly noted the advantages of free reproduction in the
month ``Art Bonus'' came into force (prior to the approval of the restrictive amendment): one could observe not only that
the images did not damage the archival material, but rather, being subject to less movement, actually contributed to
its conservation.\footnote{\url{https://fotoliberebbcc.wordpress.com/2015/06/06/4724/}}

If the remainder of the problem is indeed conservation, it is difficult to understand why this might be considered a
problem only in Italy: the policies of the National Archives of the United Kingdom and the French Archives Nationales
already foresee free reproduction with a user’s personal device. In recent months the British Library has spread across
the web a short video demonstration in order to instruct the users in the proper handling of the codex manuscript.
Specifically, the video prompts the user to employ simple lead cordons to hold open the pages of a codex without
compressing the back of the volume and risking damaging the
binding.\footnote{\url{http://www.bl.uk/reshelp/inrrooms/stp/copy/selfsrvcopy/book\_photography\_video.mp4}.}

In addition, digital replaces contact photocopy (which still exists in some archives), primarily by reducing the wear
and tear in handling a manuscript over the course of extended periods. As such, reproduction at a distance not only
does not damage media more than normal handling during consultation, but rather may be a powerful ally to its
conservation. For this reason it should be incentivised and not limited or in fact denied as happens today. 

To those fearing `abuse' relating to improper use of images (different from those permitted by authorisation) it is easy
to respond that prior authorisation had already been eliminated from ``Art Bonus'' for photographs of all cultural
heritage other than bibliographical or archival records. In other words, `ex post' control instead of `ex ante'
control, characterised as the only genuinely effective measure. It is hard to imagine now, in the digital era, strict
ironclad control over all who request daily permission to reproduce mass of images which are each day authorised only
for ``educational purposes''! This is a problem which doesn’t exist in France: the Archives Nationales does not in fact
provide any specific request for autonomous reproduction, while the Bibliothéque National de France allows anyone to
freely download PDF documents or antique volumes in good resolution without the need to include distinctive watermarks.

The application of tariffs for reproduction and reuse of cultural heritage images appears, even more than a means of
generating profits, a pretext designed to discourage reproduction of manuscripts and ancient volumes and to minimise
the risk of abuse associated with using the images for profit-making activity, in a way which might `damage' the public
treasury.\footnote{\url{http://www.bianchibandinelli.it/2015/05/25/fotografie-libere-un-comunicato-dellassociazione-bianchi-bandinelli/}}




\section{The proposal of ``Fotografie libere per i Beni Culturali''}




To meet scholars’ demands, in September 2014 researchers spontaneously formed a movement of ideas, ``Fotografie libere
per i Beni Culturali''. The group launched a petition asking Minister of Culture on. Dario Franceschini for free
reproduction in archives and libraries, renewing the original spirit of the ``Art Bonus''
decree.\footnote{\url{https://fotoliberebbcc.wordpress.com/category/adesioni-e-contatti/}} It was an unprecedented
initiative given not only the number of signatures (more than 4200), but also the quality of subscribers: a chorus
comprising the highest representatives from each of the historical-humanitarian disciplines from around Europe:
associations, university docents, researchers, students, and simple fans of local history, along with directors of
state archives and functionaries of the ministry
itself.\footnote{\url{http://www.ilgiornaledellarte.com/articoli/2015/4/123892.html}.}

Moreover, the movement produced a proposal to amend article 108 of the Codice dei Beni Culturali regulating the
reproduction of cultural heritage,\footnote{\url{https://fotoliberebbcc.wordpress.com/category/la-nostra-proposta/}} to eliminate the
exclusion of printed books and documents from the liberalisation archive, hence complying with copyright and personal
privacy for archival documents.
\newpage
Free photography is therefore a fundamental tool which, without contradicting the demands of conservation, greatly
facilitates the task of transcribing documents and historical research – allowing significant time and money savings
for the scholar, especially those forced to move to archives distant from their own place of residence.

Free photography means the ability to use one’s own smartphone or camera in archives and libraries during consultation,
without the need to request prior written approval (for purposes other than profit). A `cultural’ exception should be
provided for a particular form of profit, scientific publishing, given the belief that research is meaningful only if
it can be disseminated in the broadest way.\footnote{Circolare Mibact 21/2005.} ``Fotografie libere per i Beni
Culturali'' in fact proposes to make free the publication of images of cultural heritage in scientific texts in
circulation and of a limited cover price (below 77 Euro and 2000 copies) with the sole duty to always specify the name
of the library/archive of provenance, and to provide a copy of the publication. In such cases the official request for
authorisation by ordinary post may be substituted by a simple communication online at the institute with the intent to
publish. For the ``greater'' publications, other than the usual indicated above, and in general for moneymaking
activities, the concession regime will remain in force, requiring both a formal authorisation request as well as a
payment for the canon of use (royalties).






\section{The free reuse cultural heritage images}




The reuse of cultural heritage images, briefly mentioned in the proposal scope for ``Fotografie libere per i Beni
Culturali'', intersects closely with EAGLE’s need to publish Latin and Greek inscriptions online for cultural purposes.
To this end, EAGLE established an agreement with the Italian Ministry of Culture on the 21st November 2005, to publish
on the web low-resolution reproductions of previously published Latin and Greek inscriptions dated before the 7th
century. A second agreement, signed in 2012 with the Archivio di Stato di Roma (State Archives of Rome), authorizes
EAGLE to web-publish images of archival documents, including dates and information of interest for the study of antique
epigraphy. Despite these formal agreements, those working for EAGLE in Italy and across Europe, understand how
difficult it is to publish images of inscription – mostly due to the mistrust of those who still view diffusion of
digital images as a form of ``expropriation'' and not an occasion of cultural enrichment.

The added-value of digital resides in its ability to disseminate knowledge, and not in static storage, as clearly
demonstrated in numerous international examples which head in the direction of an ever-greater open majority toward
free-reuse: the British Library for instance, which incidentally in the past months has liberalised reproductions to
satisfy the demands of users, explicitly promotes the sharing of images and manuscripts on social networks, recognising
the `great benefit in sharing images’. An interesting case is the Metropolitan Museum of New York, which has made
available hundreds of thousands of high-resolution photographs of their own works, even allowing their reuse in
scientific publications, in free format but also free from prior authorisation; the Walters Art Museum in Baltimore
proposes an even more extreme model in giving anyone the possibility to reuse excellent resolution images for any
purpose, including commercial (!) with a ``CC0'' license.

Free reuse of the images, at least for cultural purposes, has obvious positive effects on the liberal circulation of
scientific content, and also represents an important way to improve the visibility of institutions and their
collections. Museums, archives and libraries should therefore be seen as active and dynamic centres of cultural
promotion, more so than bureaucratic offices for the static conservation of cultural heritage.\footnote{Cfr. on the
Creative Commons licences in archaeology: \citet{brugnoli_fotografia_2013}; \citet{gualandi}.}




\section{Encouraging first results and European perspectives for project EAGLE}



The effort underlying the collection of thousands of signatures has already led to encouraging openings. A recent
parliamentary question to the minister on whether to restore the original sprit of the ``Art Bonus''
decree,\footnote{\url{http://www.senato.it/japp/bgt/showdoc/showText?tipodoc=Sindisp\&leg=17\&id=914146}} and the
comparable availability shown by the Ministry of Culture, are in fact events which inspire cautious optimism –
primarily because they mark a return to this topic in parliament after one year. It begins a course that will hopefully
soon arrive at the modification of article 108 of the ``Codice dei Beni Culturali'' regulating reproduction and the
release of an appropriate circular which should affirm, without ambiguity:


\begin{enumerate}
\item the free reproduction in archives and libraries for research purposes and in accordance with complete respect for
privacy and copyright law;

\item the free reuse of images of cultural heritage for purposes outside of financial gain, or free publication online
and in print within the limits mentioned above.\footnote{Cfr. n. 10.}

\end{enumerate}

It is important that these principles remain valid in archives and libraries which pertain not only to the State, but
also to the local administrations and – why not – to the private owners of cultural heritage which, although private,
constitute the patrimony of public interest. If in Italy one might achieve a similar result, as is hoped, the immediate
next objective would be to export this model to other countries in the EU to encourage the widest possible circulation
of information in the field of historical research at the community level.

Photography, as noted here, is an indispensable method for studying not only epigraphic text but also the supportive
material which contains it. The advantage for EAGLE will then be two-fold: the liberalisation of photography in
archives and libraries can facilitate the study of numerous epigraphic repertoires which can be found in manuscripts
and antiquarian sources, while the free reuse of cultural images in museums across Europe may allow the broadest
sharing of inscription reproductions on the web.

\nocite{brugnoli_riproduzione_2013}
\nocite{casini_gli_2015}
\nocite{ciociola_libere_2015}
\nocite{gallo_il_2014}
\nocite{giunta_viva_2015}
\nocite{lupoli_libere_2015}
\nocite{manacorda_fotografare_2014}
\nocite{manacorda_tutela._2014}
\nocite{modolo_ricerca_2015}
\nocite{pavolini_situazione_2015}
\nocite{pigliaru_selfie_2015}
\nocite{stella_gabella_2014}
\nocite{stella_costose_2015}
\nocite{stella_biblioteche_2015}
\nocite{tumicelli_questione_2014}
\nocite{volpe_limmagine_2014}
\nocite{_lettera_2015}
\nocite{_appello_2013}

\bibliographystyle{sapauth-eng}
\bibliography{../../EAGLE}

\end{document}