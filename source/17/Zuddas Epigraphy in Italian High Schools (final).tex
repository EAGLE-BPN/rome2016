% !TEX encoding = UTF-8 Unicode
% !TEX program = pdflatex
% !TEX spellcheck = en_US


% In order to correctly compile this document,
% execute the following commands:
% 1. pdflatex
% 2. pdflatex
% 3. pdflatex



\documentclass[amsthm,ebook]{saparticle}

% IF YOU USE PDFLATEX
\usepackage[utf8x]{inputenc}
% if you write in english and in greek
\usepackage{ucs}
\usepackage[greek,english]{babel}
\languageattribute{greek}{polutoniko}

% IF YOU USE XELATEX
%\usepackage{polyglossia}
% if you write in italian
%\setmainlanguage{italian}
% If you want put some ancient greek:
%\setotherlanguage[variant=polytonic]{greek}
%\newfontfamily{\greekfont}[Ligatures=TeX]{Palatino Linotype}

% dummy text (remove in a normal thesis)
% remove if not necessary
\usepackage{siunitx}
%Natbib for bibliography management
\usepackage[authoryear]{natbib}
% custom commands
\newcommand{\bs}{\textbackslash}

%%%%%%%%
%TITLE:%
%%%%%%%%
\title{}
\author{Utente}
\date{2015-11-16}
\begin{document}
Abstract

\begin{figure}
\centering
\begin{minipage}{10.901cm}
1.\ \ EPIGRAPHY IN ITALIAN HIGH SCHOOLS\newline


Enrico Zuddas

University of Perugia

enricozudd@yahoo.com
\end{minipage}
\end{figure}
The paper focuses on the possible uses of Epigraphy when teaching Latin to Italian High School students. Inscriptions
(especially from the school’s territory) make Latin “come to life” because even the simplest of texts is real.
Inscriptions are an excellent tool to internalize declensions, but they can also broaden the horizons offered by the
literary sources. The Eagle databases offer ready to use materials to teachers who are less familiar with the
traditional supports.

Keywords

High School Education, Latin Grammar, Lexicon, Translations, EDR Database, Roman Asisium.


\bigskip

1.1. Introduction

High school students often feel that Latin is not a “real” language. Every attempt to use Latin as a modern language, by
the creation of words from the contemporary world or by using it in a conversation, makes Latin seem even more
artificial.

In this framework, for a Latin teacher the first two years of high school are the most challenging. This is when
students have to acquire the grammar, but hardly see the point of learning so many rules by heart (even if the
prescriptive approaches have now given way to descriptive linguistics). However, when students start studying
Literature and reading literary texts, they generally realize that it was worth making the effort. Hence, one can
imagine the frustration of Liceo Linguistico (Foreign Languages High School) pupils, who only do Latin for the first
two years, and will never see the results of such hard work.

One could object that in the first two years teachers should mix grammar and culture: but with only two lessons per
week, and classes with a high number of students, there would simply be not enough time. Moreover, the approach to
unedited literary texts is arduous because they contain too many unknown structures and often need to be at least
partially translated.

Inscriptions can be an effective solution to this problem: they make Latin authentic. They are often short and therefore
quite easily readable. They exist in reality, they can be seen and even touched – which should never be forgotten in a
society where visual communication is so important. Through epigraphy the Classics become, in a way, “multimedia”: this
is what scholars mean by the expression “words on stone”. Even if generally, when visiting a museum, inscriptions are
not of such interest as the statues or mosaics, they do strike students as something concrete and, in a certain sense,
“alive”. That is why they are a good tool to improve the ability to read, understand and translate Latin.

The use of local materials, which can be checked out personally, may more easily arouse the learners’ attention. That is
why in this paper I will mainly refer to inscriptions from modern Umbria (regiones VI and VII, from Perusia) and
especially from Asisium, used at school during my experimental lessons. The local museum, best known as the “Foro
romano”, hosts a huge collection of inscriptions that were catalogued in 2008 by pupils of the Liceo Properzio, under
my supervision, with details of the type of inscription, material, place of origin and chronology. Previously a booklet
about Roman Assisi had been
written\footnote{http://www.liceoassisi.it/index.php?option=com\_content\&view=article\&id=47:assisi-romana\&catid=6:i-nostri-risultati\&Itemid=34.},
both in Italian and English, comprising a page about the famous tetrastyle with a translation of the inscription on the
base (CIL 11, 05372 = EDR025323, later inserted in the MediaWiki page), which recalls the official inauguration of the
aedicule: at the time, as was the Roman custom, money was given to decurions, seviri Augustales and the common people.
Thanks to these few lines, the class is taken on a journey to ancient Asisium, and gets a lot of feedback about
religion, architecture, social life and economy in the first century A.D.

1.2. How to use inscriptions

Inscriptions in high school can be used in a variety of ways, but particularly in the following fields:

\begin{itemize}
\item elements of Latin grammar, syntax and linguistics\footnote{ Hartnett 2012; McCarthy 1992.};
\item culture and history;
\item and also, to a lesser extent, lexicon. 
\end{itemize}

\bigskip

1.2.1. Lexicon in context

Although the acquisition of lexicon is essential in order to access the conceptual categories of a culture, it is
extremely difficult to learn words without using them actively, and it is a terrible mistake, often made by school
textbooks, to provide long lists of terms, especially if de-contextualized and only based on the frequency with which
they are used.

Many manuals, influenced by modern languages, offer lexicon in concrete fields (such as food, clothing or education),
but this does not necessarily entail that the students will be more attracted by these topics. There is also the risk
of reducing culture to anecdotes. Experience shows that with a limited amount of time and taking into account the
selective memory of teenagers, you have to choose what your priority is. The main reason why a school-level student
should do Latin is because of the deep impact of Roman cultural and linguistic heritage on our world: therefore, in my
opinion, the words that students need to learn the most are:

\begin{itemize}
\item the ones that are relevant for their Italian derivates (e.g. os, oris);
\item the ones that belong to the most significant semantic fields (always keeping in mind that some words are more
important than others: for instance, knowing the difference between bellum and pugna helps to develop the comprehension
of the two different categories; on the contrary, a non-specialized student does not need to know who a primus pilus
was);
\item the abstract ones that are fundamental to understanding the Roman way of thinking (e.g. imperium, virtus, fas).
\end{itemize}
Accordingly, a teacher should concentrate on those aspects of Roman society that are still pre-eminent for us:
archaeology (Roman buildings, domus, theatres, roads), myths and religion, politics (Empire, war and globalization).
Inscriptions certainly offer a less varied lexical repertoire than a literary text, but can still help to give
substance to these contexts.

In an inscription such as CIL 11, 05400 = EDR025350:

P. Decimius P. l. Eros / Merula, medicus / clinicus, chirurgus, / ocularius, VIvir. / Hic pro libertate dedit
((sestertium)) (quinquaginta milia). / Hic pro seviratu in rem p(ublicam) / dedit ((sestertium)) (duo milia). / Hic in
statuas ponendas in / aedem Herculis dedit ((sestertium)) (triginta milia). / Hic in vias sternendas in / publicum
dedit ((sestertium)) (triginta septem milia). / Hic pridie quam mortuus est / reliquit patrimoni ((sestertium))... 

not only does a student make contact with the technical lexicon of medicine and words related to building activities
(statuas ponere, vias sternere), but he can also perceive the “evergetic spirit” of an ancient society: a physician was
usually a freedman coming from the East, who could be very rich and spend his money for public utility; whenever a
person obtained a priesthood, he used to pay a summa honoraria. The text can also be used for linguistic purposes
(prepositions like pro; partitive genitive; deponent verbs).

This does not mean that inscriptions are not useful for seeing specific abstract words in their context: when talking
about Roman virtues and mos maiorum, the famous Clipeus from Arles (\emph{AE} 1952, 0165 = \emph{AE} 1994, 0227) is a
perfect complement to chapter 34 of the Res Gestae (formally an inscription, too): \emph{Senatus / populusque Romanus /
imp(eratori) Caesari / Divi f(ilio) / Augusto / co(n)s(uli) VIII dedit clupeum / virtutis, clementiae, / iustitiae,
pietatis erga / deos patriamque.}


\bigskip

1.2.2. Grammar

One of the biggest problems in teaching Latin to young students is that they find it really difficult to understand
declensions. In the preliminary lessons the sentences used are very simple and can be understood and even translated
without really acquiring the syntactic function of the words.

For instance, if I say Iulia rosas amat, even an Italian who knows no Latin can get the meaning of the sentence.
Problems are encountered when the sentences and texts to translate become more complex: only when it is too late does
an ill-prepared teacher realize that the class has not internalized the language system and the patterns.

In funerary inscriptions the value of cases is essential. The texts can be simple, but if you want to understand who is
dead, who is the dedicant and what the relationship between them is, you have to distinguish dative and nominative
cases. Moreover, the concordance of the different elements of the onomastics (especially praenomina and filiation) is
good way to encourage (and, for a teacher, to test) the learning of declensions not in a merely mnemonic, but also in
an active way.

Examples:

\textwide{\FilledBigTriangleRight} CIL 11, 05501 = EDR025449: Noniae Privatae, C. Propertius -{}-{}-{}-{}-{}-

You can use an inscription as easy as this one to revise first and second declensions, and at the same time to teach the
structure of a Roman name.

\textwide{\FilledBigTriangleRight} CIL 11, 05461 = EDR025411: A(n)noru(m) XIX. Calventia C(ai) f(ilia) Polla, L(ucius)
Vistinius vir, Gavia mater posuer(unt). 

This inscription offers the chance to get feedback about mistakes in the use of language, or to reflect on mortality and
marriage.

\textwide{\FilledBigTriangleRight} CIL 11, 05399 = EDR025349: P(ublius) Decimius P(ubli) l(ibertus) Eros Merula VIvir
viam a cisterna ad domum L(uci) Muti stravit ea pecunia -{}-{}-{}-{}-{}- 

With an inscription like this you can address different topics, both historical (the importance of freedmen, the
imperial cult) and linguistic ones (indirect complements, uses of is ea id as an adjective and to introduce a relative
clause that, in this case, has evidently been lost).


\bigskip

At the beginner’s level, the sentences used for examples, exercises and translations are fictitious and often banal.
They give a false image of antiquity: for instance, when you start with the first declension you get the wrong
impression of a “female” world simply because the male nouns are rare. Students are unimpressed; these sentences have
no significance for them. Those provided by inscriptions are equally easy and short, but they are not banal because
they are a mirror of a society, they offer a historical perspective; there is always a story behind them. Even the
simplest ones can give us precious information.

Let us consider a famous inscription from the Cathedral in Assisi (CIL 11, 05390 = EDR025340):

Post(umus) Mimesius C(ai) f(ilius), T(itus) Mimesius Sert(oris) f(ilius), Ner(o) Capidas C(ai) f(ilius) Ruf(- - -),
Ner(o) Babrius T(iti) f(ilius), C(aius) Capidas T(iti) f(ilius) C(ai) n(epos), V(ibius) Voisienus T(iti) f(ilius)
marones murum ab fornice ad circum et fornicem cisternamq(ue) d(e) s(enatus) s(ententia) faciundum coiravere.

The comprehension of the text is easy, especially with the abbreviations solved. Even so, many considerations can be
made:

\begin{itemize}
\item for grammar: words of third declensions such as maro and fornix; complements of direction; gerundive to express
purpose; use of the form {}-ere in the perfect tense personal endings;
\item for history: the use of the Latin language prior to the Social War, as proof of the intense Romanization of the
area at the end of the II century B.C.\footnote{ Coarelli 1991.}; the presence of elements in the names that are not
Roman but of Umbrian origin;
\item for archaeology: the building of terraces to create public spaces in a town like Assisi established on a hill; the
identification of the area around San Rufino as the “acropolis”; the incorporation of the Roman wall into the left nave
of the church (the inscription being still in situ).
\end{itemize}

\bigskip

Schoolbooks rarely present these opportunities offered by inscriptions, so the main problem for a teacher is to access
appropriate material\footnote{ See the observations made by Carpenter 2006.}. Holding a PhD in Late Antiquity and
Umbrian inscriptions, I am fortunate enough to already know many sources. However, a graduate may not have enough
knowledge in epigraphy, may not be familiar with the Corpus Inscriptionum Latinarum; that is why the EAGLE project can
also offer tools for further insights.

Here are some examples taken from manuals where you can see the different attitude of authors towards the epigraphic
material:

\textwide{\FilledBigTriangleRight} Barbieri 2015, 39: the inscription of the architect C. Vettius Gratus (CIL X 3392) is
merely a decorative element on the page, without any relation to the topic (phonetic changes from Latin to Italian);

\textwide{\FilledBigTriangleRight} Domenici 2012, 40: programmata from Pompeii are introduced to explain the Roman
naming system;

\textwide{\FilledBigTriangleRight} Gambis Meini, Roffi, Guidotti Bacci 2013, 269-270: tabellae defixionum are used for
different purposes (demonstrative adjectives, functions of subjunctive); the subject (magic in the ancient world) may
not be relevant in a school context (the possession of such knowledge is not required), but sounds very intriguing to
students.


\bigskip


\bigskip


\bigskip

1.3. Translations

Translating inscriptions is a difficult task for everyone and especially for high school students.

First of all, they will not find any help elsewhere: every other Latin text can easily be found – even if not always
correctly translated – on student internet sites and in blogs (such as Splash Latino). But we all know very well that
the Internet still lacks many Italian translations of inscriptions.

Secondly, the style and the structures are different. Even when all abbreviations and integrations are explained (a
teacher should at least show the meaning of round and square brackets but should not ask a pupil to solve an
abbreviation, except the easy ones), the word order cannot immediately be reconstructed, especially in decrees and
carmina epigraphica. School dictionaries are not intended for interpreting epigraphical lexicon: some words may not be
present, their meaning may not always be explained. A lot of institutions and formulae which are clear to a specialist
(e.g. quattuorvir iure dicundo) may be hard to understand or to translate. It is, however, also the case that a teacher
cannot dedicate too much time to introducing these words to the class because his ultimate aim is different; so it is
better to focus on materials that do not contain much specific lexicon.

Examples:

\textwide{\FilledBigTriangleRight} CIL 11, 04431 = EDR025160: [Inf]austo, levis umbra, tuo mihi flebilis hora / sorte
tua certe tempus in omne fuit.

This funerary inscription from Ameria contains two verses, but the position of the words is tricky; infausto may be
taken as an adjective (as it is most commonly) and its meaning is basically the same as sorte tua; students who are not
used to poetry may not recognize the anastrophe tempus in omne.

\textwide{\FilledBigTriangleRight} CIL 11, 04391 = EDR025123: Iuliae M(arci) f(iliae) Felicitati, / uxori C(ai) Curiati
Eutychetis / IIIIvir(i), magistrae Fortu/nae Mel(ioris), coll(egium) centonarior(um) / ob merita eius. Quo honore /
contenta sumptum omnem / remisit et ob dedic(ationem) ded(it) sin/gulis ((sestertios)) XX n(ummos) et hoc amplius /
arkae eorum intul(it) ((sestertium)) V m(ilia) n(ummum) / ut die natalis sui (ante diem) V Id(us) Mai(as) / ex usuris
eius summae epu/lantes imperpetuum divider(ent), / quod si divisio die s(upra) s(cripta) celebrata non / fuerit tunc
pertineb(it) omn(is) summa / ad familiam publicam. 

Specialists are well used to an inscription like this. A high school student may find unexpected difficulties in
understanding the meaning of magistra (not teacher but priestess), but also centonarius or even hoc amplius; not to
mention the Roman calendar system, which always takes too much time to explain!

Nevertheless, this kind of challenge is exactly what makes inscriptions the perfect tool to fully appreciate what the
art of translation is.

The article published by F. Bigi in the Proceedings of the First EAGLE International Conference\footnote{ Bigi 2014.}
includes many observations regarding the problems that may be encountered, for instance when translating names and
titles. In particular, I find the suggestion that round brackets should be used in the translations to provide further
explanations about specific offices rendered with the technical derivative word, or for concepts omitted in the
original Latin text, very useful. Examples of this could be words such as centonarii or marones seen in inscriptions
mentioned above, but also the following ones:

\textwide{\FilledBigTriangleRight} CIL 11, 04213 = EDR130908: Interamna Nahars should be further qualified as “Terni” to
help those readers who are not familiar with Umbrian cities;

\textwide{\FilledBigTriangleRight} CIL 11, 01925 = EDR142701: the names of the emperors M. Aurelius Antoninus and M.
Antoninus Pius Germaticus Sarmaticus need to be explained (Caracalla and Marcus Aurelius) to avoid confusion. It should
be noted that in my classes proper names, and not only those of emperors but also of other people, were generally
translated into Italian, even if it is advisable to transcribe them in the nominative case\footnote{
http://www.eagle-network.eu/wiki/index.php/Guidelines\_for\_Translators.}; the same thing was done with cognomina ex
virtute, considering that they are intuitively interpretable for an Italian.


\bigskip

This type of activity works better with Liceo Classico (Grammar School) pupils, who do translations from Latin and Greek
almost every day and who are more at ease with the use of dictionaries. However, there are still hurdles to overcome.
The ministerial syllabus set out for the course focuses on Literature and culminates in a specific exam requirement,
the translation of a piece of literary prose. Is this “epigraphical” activity helpful? Does it take up too much of the
time which should be employed in translating the Classics? The answer to both these questions is “yes”. On the one
hand, as I said, students have to “jump into translating” without a net (the Net, in fact). On the other hand, if at
the end of the final year pupils are required to translate a passage from certain authors, then clearly it would be
more appropriate for them to concentrate on this activity as much as possible during the months prior to the exam. For
this reason, epigraphy can only be a supplement to traditional assignments; the Italian national curricula are
apparently very free, but at the same time they are very rigid. Yet, a few forays into epigraphy can be stimulating,
because the class perceives them as an intriguing novelty, especially if not subject to assessment. After all, it would
probably be too difficult to prepare a test with grades and scores on this subject and could deprive this activity of
its extemporaneous and enjoyable aspect.

1.4. Inscriptions and Literature: a few samples

On a more advanced level, inscriptions may also integrate certain aspects related to the study of Latin Literature. The
most typical example could be a comparison between the Tabula Claudiana (CIL 13, 01668) and Tacitus’ account (Annales
XI, 23-24)\footnote{ On which see Jahn 1993.}: reading the original document is a privileged occasion to determine how
reliable the historian is when using his sources.

Highly original suggestions have been provided by Mauro Reali, who is also the author of different school manuals, in a
paper published online\footnote{http://mediaclassica.loescher.it/nuove-e-\%93vecchie\%94-forme-di-multimedialita.n2799
; see also Reali, Turazza 2015.}. Being an expert on the subject, he offers a comparison between the “noble” form of
the political-philosophical amicitia presented in Cicero’s Laelius and the term \emph{amicus} mentioned in inscriptions
from the lower levels of society, such as CIL 05, 05300 from \emph{Comum} (a funerary stela made by a freedman for
Pliny the Younger) or CIL 05, 05923 = EDR124245 regarding a strange case of “friend deletion” long before the Facebook
era\footnote{ http://www.laricerca.loescher.it/lingue-classiche/327-un-amico-o-amicus-e-per-sempre.html.}.

An interesting example from Umbria can be found in AE 1992, 0560-0561 = EDR150769 and EDR150784. The first gravestone
recalls the acquisition of a tomb - which had previously been despoiled - by an heir of the founder, who then installed
another cippus for 40 friends (amicis meis, i.e. freedmen probably belonging to the same association):

Viator, resiste et rogo / te et lege. Post annos XXVII ven[i] / Hispellum, in patriam meam. Scio / me oportere colere
hunc locum / ubi ossa meorum requiescunt et mea / et amicorum meorum. Ex hoc sepulch[ro] / cippi perierunt duo et
frontes duae sciun[t] / qui surupuit et acturi simus et legimus, / satis est testium etqs.

As for the Augustan age, Reali suggests showing some monumental inscriptions of the princeps (he impressively goes so
far as to compare the qualifications \emph{Imperator Caesar Augustus} to a modern logo or even a \emph{hashtag}). On
this subject, the altars Augusto sacrum put up by Perusia restituta (CIL 11, 01923 = EDR142666, EDR142667, EDR142668,
EDR142669), even with a simple text, offer the opportunity to deal with an aspect as crucial as the imperial cult.
Traditionally, we read that Augustus was worshipped directly only in the Eastern provinces, but not in Rome and Italy.
The inscriptions from Perusia testify that things are effectively more complex; these documents also offer remarkable
information about the restitutio of the town, destroyed at the end of the bellum in 40 B.C. A few years later Perusia
would become Augusta, as you can read on the city gates, especially on the newly restored Etruscan Arch (CIL 11, 01929
= EDR142706).

Teaching, as I do, in a school named after Propertius, I always stress the importance of reconstructing the origin of
the poet through the epigraphical data of the gens Propertia: the greatest number of written documents of the family
having been found in Assisi. For more than two hundred years, beginning with the Vois(ienus) Ner. (filius) Propertius
mentioned among the Umbrian magistrates of the late II century B.C., this gens stands out in the town for its social
influence and wealth\footnote{ Forni 1986; Zuddas 2006.}. It is always very exciting to combine the information on
Passennus Paullus Propertius Blaesus given by CIL 11, 05405 = EDR025355 and that contained in two Letters by Pliny (VI,
15 and IX, 22). Pliny, showing great concern for his friend’s illness but also great esteem for him as an elegiac poet,
asserts that he is a descendant and fellow citizen of Propertius; the inscription (the front of an honorific base),
providing the full name, with the tribe Sergia typical of the inhabitants of Asisium, is indirect, but clear, evidence
that the Augustan poet was born there. The information provided by the inscription and the literary text integrates
perfectly. Students find it fascinating to look for traces of the poet inside the town, especially when they read the
following graffito on an interior wall of a Roman house underneath the church of Santa Maria Maggiore (EDR028769):

[{}- - - I]o[323?]vino consulibb(us) (ante diem) VIII Kal(endas) Martias domum oscilavi Musae. 

This house was still visited in the fourth century A.D., which really supports the theory that it used to be the poet’s
residence, and remained an object of reverence for centuries\footnote{ Boldrighini 2014, 244-246.}. Through the stories
of Passennus Paullus and Sextus Propertius macro-history and local history meet to make the past come alive in every
corner of the modern town.

\bibliographystyle{sapauth-eng}
\bibliography{../../EAGLE}

\end{document}