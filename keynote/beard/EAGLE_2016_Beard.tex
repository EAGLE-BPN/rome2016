% !TEX encoding = UTF-8 Unicode
% !TEX program = pdflatex
% !TEX spellcheck = en_US


% In order to correctly compile this document,
% execute the following commands:
% 1. pdflatex
% 2. pdflatex
% 3. pdflatex



\documentclass[amsthm,ebook]{saparticle}

% IF YOU USE PDFLATEX
\usepackage[utf8x]{inputenc}
% if you write in english and in greek
\usepackage{ucs}
\usepackage[greek,english]{babel}
\languageattribute{greek}{polutoniko}

% IF YOU USE XELATEX
%\usepackage{polyglossia}
% if you write in italian
%\setmainlanguage{italian}
% If you want put some ancient greek:
%\setotherlanguage[variant=polytonic]{greek}
%\newfontfamily{\greekfont}[Ligatures=TeX]{Palatino Linotype}

% dummy text (remove in a normal thesis)
% remove if not necessary
\usepackage{siunitx}
%Natbib for bibliography management
\usepackage[authoryear]{natbib}
% custom commands
\newcommand{\bs}{\textbackslash}

%%%%%%%%
%TITLE:%
%%%%%%%%
\title{Putting ancient inscriptions in the limelight}
\author[]{Mary Beard\corref{first}}

\begin{document}
\maketitle
\begin{abstract}
This lecture will reflect on public engagement with ancient inscriptions: what is it about inscriptions that interests a wider audience, and -- just as important -- what puts them off? I shall be drawing on a series of BBC documentaries, ``Meet the Romans'' (``Ti presento i Romani'') which used Roman epitaphs and other inscribed texts as a way into the life of ordinary Romans, but will reflect more widely on how academics and museum professionals can (and already do) make inscriptions come alive for the public.
\end{abstract}

Mary Beard is one of Britain’s best-known Classicists – a distinguished Professor of Classics at the University of Cambridge and Fellow of Newnham College. She has written numerous books on the Ancient World, including the 2008 Wolfson Prize-winner, Pompeii: The Life of a Roman Town, The Roman Triumph, Classical Art from Greece to Rome, as well as popular books on the Parthenon and Colosseum. In addition she has presented a highly-acclaimed TV series, Meet the Roman as well as documentaries about Pompeii and the Emperor Caligula. Mary’s interests range from the social and cultural life of the Ancient World to Victorian understanding of antiquity. Mary is also Classics editor of the Times Literary Supplement and writes an engaging, and thought-provoking, blog, A Don’s Life. Mary’s academic achievements were acknowledged, in 2010, by the British Academy which elected her as a Fellow and she was made an OBE in the New Year’s Honours List 2013 for services to Classical scholarship. Her latest book is on the subject of Roman Laughter and her forthcoming history of Rome, SPQR, will be published in Autumn 2015.

\end{document}


