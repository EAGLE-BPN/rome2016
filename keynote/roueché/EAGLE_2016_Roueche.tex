% !TEX encoding = UTF-8 Unicode
% !TEX program = pdflatex
% !TEX spellcheck = en_US


% In order to correctly compile this document,
% execute the following commands:
% 1. pdflatex
% 2. pdflatex
% 3. pdflatex



\documentclass[amsthm,ebook]{saparticle}

% IF YOU USE PDFLATEX
\usepackage[utf8x]{inputenc}
% if you write in english and in greek
\usepackage{ucs}
\usepackage[greek,english]{babel}
\languageattribute{greek}{polutoniko}

% IF YOU USE XELATEX
%\usepackage{polyglossia}
% if you write in italian
%\setmainlanguage{italian}
% If you want put some ancient greek:
%\setotherlanguage[variant=polytonic]{greek}
%\newfontfamily{\greekfont}[Ligatures=TeX]{Palatino Linotype}

% dummy text (remove in a normal thesis)
% remove if not necessary
\usepackage{siunitx}
%Natbib for bibliography management
\usepackage[authoryear]{natbib}
% custom commands
\newcommand{\bs}{\textbackslash}

%%%%%%%%
%TITLE:%
%%%%%%%%
\title{Where Eagles Dare}
\author[]{Charlotte Roueché\corref{first}}
\begin{document}
\maketitle
\begin{abstract}
The publication of inscriptions in digital form has been evolving since the 1980s, and online publication since the 1990s. It is useful to see where EAGLE is positioned in this evolution – a huge step forward, but not the end of the road. What can we learn from our progress so far? What have the main challenges been? And what does this suggest about the next stages?

\end{abstract}

Charlotte Roueché studied Classics at Cambridge, and Byzantine Studies in Paris and Washington; as a pupil of Joyce Reynolds, she became involved in publishing Greek inscriptions, with a particular focus on material from Late Antiquity. In 1984 she was appointed to King’s College London, where an important centre of Digital Humanities started to develop in the 1990s; this encouraged her to explore the possibilities for publishing inscribed texts online. In 1999 Tom Elliott published the first proposal for EpiDoc; working with him and Gabriel Bodard, Charlotte was able to use EpiDoc to publish several online corpora: \url{http://insaph.kcl.ac.uk/ala2004, http://insaph.kcl.ac.uk/iaph2007}, and \url{http://irt.kcl.ac.uk/irt2009}. At present she is preparing a digital corpus of inscriptions of Roman Cyrenaica, and developing plans for a shared portal, Inscriptions of Libya. She has also recently published an eleventh century Byzantine text online, in a project exploring the uses of RDF to analyse intertextuality: \url{http://ancientwisdoms.ac.uk}.

\end{document}


