% !TEX encoding = UTF-8 Unicode
% !TEX program = pdflatex
% !TEX spellcheck = en_US


% In order to correctly compile this document,
% execute the following commands:
% 1. pdflatex
% 2. pdflatex
% 3. pdflatex



\documentclass[amsthm,ebook]{saparticle}

% IF YOU USE PDFLATEX
\usepackage[utf8x]{inputenc}
% if you write in english and in greek
\usepackage{ucs}
\usepackage[greek,english]{babel}
\languageattribute{greek}{polutoniko}

% IF YOU USE XELATEX
%\usepackage{polyglossia}
% if you write in italian
%\setmainlanguage{italian}
% If you want put some ancient greek:
%\setotherlanguage[variant=polytonic]{greek}
%\newfontfamily{\greekfont}[Ligatures=TeX]{Palatino Linotype}

% dummy text (remove in a normal thesis)
% remove if not necessary
\usepackage{siunitx}
%Natbib for bibliography management
\usepackage[authoryear]{natbib}
% custom commands
\newcommand{\bs}{\textbackslash}

%%%%%%%%
%TITLE:%
%%%%%%%%
\title{Infrastructures for Digital Research: New opportunities and challenges}
\author[]{Lorna Hughes\corref{first}}
\begin{document}
\maketitle
\begin{abstract}
The expansion of digital cultural heritage from library, archive, museum and university collections has created tremendous opportunities for research, not least through new methods for collaboration and communication. The ‘digital turn’ has also enabled innovative approaches to representing spatial and temporal aspects of primary source materials. However, challenges still exist in accessing and using digital heritage: access to the content; availability and use of digital tools and methods; and the publication of research outputs in ways that are recognized by our peers.
 This presentation will discuss some approaches to addressing these issues, based on the theory and practice informing the development of digital research infrastructures. It will also discuss the role of the Europeana in the digital research space, and some important new developments internationally to support the use and description of digital content, tools and methods for research. 
\end{abstract}

\vspace{1cm}
Lorna M. Hughes is Professor of Digital Humanities at the University of Glasgow.Her research addresses the creation and use of digital cultural heritage for research, with a focus on collaborations between the humanities and scientific disciplines. A specialist in digital humanities methods, Hughes is the author of Digitizing Collections: Strategic Issues for the Information Manager (London: Facet, 2004), the editor of Evaluating \& Measuring the Value, Use and Impact of Digital Collections (London: Facet, 2011), and the co-editor of The Virtual Representation of the Past (London: Ashgate, 2007). She was the Chair of the European Science Foundation (ESF) Network for Digital Methods in the Arts and Humanities (\url{www.nedimah.eu}) from 2011-15, which developed the NeDiMAH Methods Ontology for the Digital Humanities (NeMO: \url{nemo.dcu.gr/}). Other notable digital projects include the AHRC-funded The Snows of Yesteryear: Narrating Extreme Weather (\url{eira.llgc.org.uk}) and the Jisc-funded digital archive, The Welsh Experience of the First World War (\url{cymruww1.llgc.org.uk}).

\end{document}